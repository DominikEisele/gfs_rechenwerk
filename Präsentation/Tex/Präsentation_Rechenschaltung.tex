\RequirePackage{etex}
\documentclass{beamer}
\usepackage[latin9]{inputenc}
\usepackage[ngerman]{babel}
\usepackage{amsmath}
\usepackage{listings}
\usepackage{color}
\usepackage{pictex}

\title[Rechenwerk]{Digitales Rechenwerk}
\author{Dominik Eisele}
\institute[WSS]{Werner-Siemens-Schule}
\date{\today}
\subject{FTP}
\keywords{FTP, File Transfer Protocol}

\usetheme{Singapore}
\usecolortheme{rose}
\usefonttheme{serif}
\setbeamertemplate{navigation symbols}{}
\setbeamertemplate{footline}[frame number]
\beamersetuncovermixins{\opaqueness<1>{25}}{\opaqueness<2->{15}}

\definecolor{dkgreen}{rgb}{0,0.6,0}
\definecolor{gray}{rgb}{0.5,0.5,0.5}
\definecolor{mauve}{rgb}{0.58,0,0.82}

\lstset{language=C}
\lstset{numbers=left,
	numberstyle=\tiny,
	numbersep=5pt,
	breaklines=true,
	showstringspaces=false,
	frame=l ,
	xleftmargin=15pt,
	xrightmargin=15pt,
	basicstyle=\ttfamily\scriptsize,
	stepnumber=1,
	keywordstyle=\color{blue},		% keyword style
	commentstyle=\color{dkgreen},	% comment style
	stringstyle=\color{mauve}		% string literal style
}

%\AtBeginDocument{\addtobeamertemplate{block begin}{\setlength\abovedisplayskip{0pt}}}
%\AtBeginDocument{\addtobeamertemplate{block begin}{\setlength\abovedisplayshortskip{0pt}}}

\let\oldsqrt\sqrt 
\def\sqrt{\mathpalette\DHLhksqrt}
\def\DHLhksqrt#1#2{\setbox0=\hbox{$#1\oldsqrt{#2\,}$}\dimen0=\ht0
\advance\dimen0-0.3\ht0
%0.3 ist das Ma� f�r die Hakenl�nge, relativ zum Inhalt der Wurzel
\setbox2=\hbox{\vrule height\ht0 depth -\dimen0}%
{\box0\lower0.4pt\box2}}

\setcounter{tocdepth}{1} 

\begin{document}

\begin{frame}
	\titlepage
\end{frame}

\begin{frame}{Inhalt}
	\tableofcontents
\end{frame}


\section{Rechnen mit dualen Zahlen}
\subsection{Rechnen mit dualen Zahlen-dots}

\begin{frame}{Rechnen mit Dualzahlen} 
	Das Rechnen mit Dualzahlen verl�uft nach den selben Rechenregeln wie das Rechnen mit Dezimalzahlen.
\end{frame}

\begin{frame}{Addition von Dualzahlen}
	Rechenbeispiel f�r eine Addition mit Dual-Zahlen.
	\begin{center}
		\begin{tabular}{crcl}
				&		 	       	 1 1 1 0 1 0 0 1 0 		 	  	\\
		 +  		&				 0 0 1 1 1 0 1 0 0			  	\\\hline
				&			      1 0 0 1 0 0 0 1 1 0			  
		\end{tabular}
	\end{center}
\end{frame}


\begin{frame}{Subtraktion von Dualzahlen}
	Rechenbeispiel f�r eine Subtraktion mit Dual-Zahlen.
	\begin{center}
		\begin{tabular}{crcl}
				&		 	       	  1 1 1 0 1 0 0 1 0 		 	  	\\
		 -  		&				  0 0 1 1 1 0 1 0 0			  	\\\hline
				&			          1 0 1 0 1 1 1 1 0			  
		\end{tabular}
	\end{center}
\end{frame}


\begin{frame}{Subtraktion mit Hilfe des Zweierkomplements}
	Da, in der Digitaltechnik, f�r die Subtraktion von Dualzahlen keine logische Verkn�pfung existiert, ist man gezwungen eine
	Subtraktion in eine Addition umwandeln.
		
	\begin{align*}
		\onslide<2->{2 - 6 &= (-4) \\} 
		\onslide<3->{2 + \left(-6\right) &= (-4) \\}  
	\end{align*}
\end{frame}


\begin{frame} 		
	\frametitle{Subtraktion mit Hilfe des Zweierkomplements}
		\hspace*{5mm} $ \underline{2 - 6 = \, ? }$ \\ \pause 
		1. Schritt: In eine Dualzahl wandeln:\\  
		\hspace*{5mm}  $ 2 - 6 \Rightarrow 10 - 110$\\ \pause
		2. Schritt: Stellen auff�llen:\\ 
		\hspace*{5mm} $ 0010 - 0110 =$ $ ? $\\\pause 
		3. Schritt: Bits negieren:\\
		\hspace*{5mm} $ 0110 \Rightarrow 1001$\\ \pause 
		4. Schritt: Hinzuaddieren von 1:\\ 
		\hspace*{5mm} $ 1001 + 0001 = 1010$\\ \pause
		5. Schritt: Minuend und Zweierkomplement addieren:\\ 
		\hspace*{5mm} $ 0010 + 1010 = 1100$\\ \pause
		6. Schritt: Ergebnis negieren:\\ 
		\hspace*{5mm} $ 100 \Rightarrow 011$\\ 
		\end{frame}
\begin{frame} 		
	\frametitle{Subtraktion mit Hilfe des Zweierkomplements}
		7. Schritt: Hinzuaddieren von 1:\\
		\hspace*{5mm} $ 011 + 001 = 100$\\ \pause 
		8. Schritt: In eine Dezimalzahl wandeln:\\ 
		\hspace*{5mm} $ 100 \Rightarrow 4$ ; da das h�chstwertigste Bit 1 ist: \\ \hspace*{5mm} Endergebnis $= -4$\\ \pause
		9. Schritt: Ergebnis:\\ 
		\hspace*{5mm} $ \underline{2 - 6 = \, (-4) }$
\end{frame}


\begin{frame}{Multiplikation von Dualzahlen}
	\begin{itemize}
		\item Es werden Produkte mit den einzelnen Stellen des Multiplikators gebildet,sie werden anschlie�end Stellenrichtig addiert. \\ \pause 
		\item Die Stellen des Multiplikators k�nnen nur Zahlenwerte zwischen Null und Eins annehmen \\ \pause
		\item $\rightarrow$ Dies kann mit einer einfachen UND-Verkn�pfung gel�st werden.
	\end{itemize}
\end{frame}


\begin{frame}{Multiplikation von Dualzahlen}
	Rechenbeispiel f�r eine Multiplikation mit Dual-Zahlen.
	\begin{center}
		\begin{tabular}{crcl}
				&		 	       	  1 0 1 1 $\times$ 1 0 1 0	 	  	\\ \hline
				&							1011000			\\
				&							000000			\\
				&							10110			\\
		 +		&							0000			  	\\ \hline
				&						        1101110		  
		\end{tabular}
	\end{center}
\end{frame}

\section{Rechenwerk}
\subsection{Umsetzung-dots}
\begin{frame}{Halbaddierer}
	\begin{columns}
		\begin{column}{0.35\linewidth}
			Ein Halbaddierer besitzt zwei Ein-, und zwei Ausg�nge. An die Eing�nge \emph{x} und \emph{y} werden jeweils die 
			Ziffern angelegt die man addieren m�chte. An dem ersten Ausgang liegt die Summe $s$ der Addition an,
			am zweiten Ausgang der �bertrag $c$.
		\end{column}
		
		\begin{column}{0.65\linewidth}
			\begin{table}[h]
				\center
				\begin{tabular}{c|c|c|c}
					\ \textbf{x} \ 	& \ \textbf{y} \		& \ \textbf{�bertrag c} \ & \ \textbf{Summe s} \ 	 	\\ \hline
					0 	& 0 		& 0          			& 0       			\\ \hline
					0 	& 1 		& 0          			& 1       			\\ \hline
					1 	& 0		& 0          			& 1      			 \\ \hline
					1	& 1 		& 1          			& 0      			 \\
				\end{tabular}
				\label{tab:halbaddierer}
			\end{table}
		\end{column}
	\end{columns}
\end{frame}


\begin{frame}{Schaltbild Halbaddierer}	
	\begin{columns}
		\begin{column}{0.5\linewidth}
			In Schaltungen wird der Halbaddierer aus zwei Bauteilen zusammengesetzt, ein Exklusiv-ODER (XOR) und ein
			UND (AND).
		\end{column}
		
		\begin{column}{0.5\linewidth}
			\begin{figure}
				\center
				\input skizze_halbaddierer
				\label{fig:halbaddierer}
			\end{figure}
		\end{column}
	\end{columns}
\end{frame}


\begin{frame}{Volladdierer}
	\begin{columns}
		\begin{column}{0.52\linewidth}
			\begin{itemize}
				\item Der Volladdierer besteht aus zwei Halbaddierern und einem ODER. \\ \pause
				\item Der Volladdierer hat einen zus�tzlichen Eingang ($c_{in}$), man kann den �bertrag aus einer vorhergegangenen Addition
					mit einbeziehen. \\ \pause
				\item Man kann mehrere Volladdierer kaskadieren um gr��ere Zahlen miteinander zu addieren.
			\end{itemize}
		\end{column}
		
		\begin{column}{0.48\linewidth}
			\begin{table}
				\begin{tabular}{c|c|c|c|c}
					\ $\mathbf{x}$ \ & \ $\mathbf{y}$ \ & \ $\mathbf{c_{in}}$ \ & \ $\mathbf{c_{out}}$ \ & \ $\mathbf{s}$ \ \\ \hline
					0          & 0          & 0            & 0             & 0                              \\ \hline
					0          & 0          & 1            & 0             & 1                              \\ \hline
					0          & 1          & 0            & 0             & 1                              \\ \hline
					0          & 1          & 1            & 1             & 0                              \\ \hline
					1          & 0          & 0            & 0             & 1                              \\ \hline
					1          & 0          & 1            & 1             & 0                              \\ \hline
					1          & 1          & 0            & 1             & 0                              \\ \hline
					1          & 1          & 1            & 1             & 1
				\end{tabular}
				\label{tab:volladdierer}
			\end{table}
		\end{column}
	\end{columns}
\end{frame}


\begin{frame}{Schaltbild Volladdierer}	
	In Schaltungen wird der Halbaddierer aus zwei Bauteilen zusammengesetzt, ein Exklusiv-ODER (XOR) und ein UND (AND).
	\begin{figure}
		\center
		\input skizze_volladdierer
		\label{fig:volladdierer}
	\end{figure}
\end{frame}


\begin{frame}{Subtrahierer}
		Beim Subtrahierer wird der Volladdierer durch einen Steuereingang erweitert. An diesem legt man f�r eine Addition eine
		$0$, und f�r eine Subtraktion eine $1$ an. Durch diesen Steuereingang wird das Zweierkomplement gebildet.
		\begin{figure}
			\input skizze_volladdierer_steuerbar
			\label{fig:steuerbarer_volladdierer}
		\end{figure}
\end{frame}


\begin{frame}{Multiplizierer}
		\begin{itemize}
		\item Der Multiplizierer kann zwei acht Bit Zahlen miteinander multiplizieren.\\ \pause
		\item Dazu	wird die Zahl an ein Schieberegister angelegt, sodass sie bei jedem Takt um eine Stelle verschoben 
		werden kann.\\ \pause
		\item Die h�chstwertigste Ziffer des Multiplikators wird abgegriffen, und mit dem Multiplikant multipliziert. \\ \pause
		\item Um die Produkte Stellenrichtig zu addieren werden die Zwischenergebnisse ebenfalls verschoben. \\ \pause
		\item Das Produkt erh�lt man, nach acht Takten Rechenzeit.
		\end{itemize}
\end{frame}


\begin{frame}{BCD-Eingabe}
		\begin{itemize}
		\item Die Eingabe erfolgt �ber ein ASCII-Feld. \\ \pause
		\item Bei einer Eingabe von einer Ziffer im Bereich 0 bis 9, liegt die Ziffer BCD codiert an den Ausg�nge $D_{0}$ bis $D_{3}$ an\\ \pause
		\item  Nach jedem Tastendruck wird die Ziffer in ein Register �bernommen, man kann an ihren Ausg�ngen die Zahl von 0 bis
			 99 abgreifen. 
		\end{itemize}
\end{frame}


\begin{frame}{Umwandlung}
		\begin{itemize}
		\item F�r die Wandlung werden zwei Z�hler und ein Wandler ben�tigt.\\ \pause
		\item Der erste Z�hler ist ein BCD-Z�hler, der zweite ein Bin�r-Z�hler.\\ \pause
		\item  Beide Z�hler haben den selben Clock.\\ \pause
		\item  Der Vergleicher vergleicht das eingegebene Signal mit dem BCD-Z�hler. \\ \pause
		\item  Sind beide Signale identisch wird der Clock unterbrochen.\\ \pause
		\item  Man kann das Bin�r-Ergebnis am Ausgang des Bin�r-Z�hlers ablesen.
		\end{itemize}
\end{frame}


\begin{frame}{7-Segment-Anzeige}
		\begin{itemize}
		\item Das Endergebnis wird Ziffer f�r Ziffer auf ein 7-Segment-Decoder gegeben.\\ \pause
		\item Die Ausg�nge des 7-Segment-Decoders werden auf die Eing�nge einer 7-Segment-Anzeige gegeben.
		\end{itemize}
\end{frame}


\section{Quellen}
\subsection{Quellen-dots}
\begin{frame}
	\begin{itemize}
		\item Junior Computer - Der einfache Einstieg in die faszinierende Computertechnik (1980)
		\item www.elektronik-kompendium.de/sites/dig/1708061.htm (11.04.2015)
		\item de.wikipedia.org/wiki/Volladdierer (16.05.2015)
	\end{itemize}
\end{frame}


\end{document}
