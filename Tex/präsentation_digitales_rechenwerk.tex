
\documentclass{beamer}
\setbeamertemplate{navigation symbols}{}


\usepackage[T1]{fontenc}
\usepackage[UTF8]{inputenc}
\usetheme{Warsaw}
\usepackage{ngerman}
\beamersetuncovermixins{\opaqueness<1>{25}}{\opaqueness<2->{15}}
\begin{document}
\title{Digitales Rechenwerk}  
\author{Dominik Eisele}
\date{\today} 

\begin{frame}
	\titlepage
\end{frame} 

\begin{frame}
	\frametitle{Inhaltsverzeichnis}\tableofcontents
\end{frame} 


\section{Rechnen mit dualen Zahlen} 
\begin{frame}
	\frametitle{Rechnen mit dualen Zahlen} 
	Das Rechnen mit Dualzahlen verläuft nach den selben Rechenregeln wie das Rechnen mit Dezimalzahlen.
\end{frame}


\subsection{Addition von Dualzahlen}
\begin{frame} 		
	\frametitle{Addition von Dualzahlen}
	Rechenbeispiel für eine Addition mit Dual-Zahlen.
	\begin{center}
		\begin{tabular}{crcl}
				&		 	       	 1 1 1 0 1 0 0 1 0 		 	  	\\
		 +  		&				 0 0 1 1 1 0 1 0 0			  	\\\hline
				&			      1 0 0 1 0 0 0 1 1 0			  
		\end{tabular}
	\end{center}
\end{frame}




\subsection{Subtraktion von Dualzahlen}
\begin{frame} 		
	\frametitle{Subtraktion von Dualzahlen}
	Rechenbeispiel für eine Subtraktion mit Dual-Zahlen.
	\begin{center}
		\begin{tabular}{crcl}
				&		 	       	  1 1 1 0 1 0 0 1 0 		 	  	\\
		 -  		&				  0 0 1 1 1 0 1 0 0			  	\\\hline
				&			          1 0 1 0 1 1 1 1 0			  
		\end{tabular}
	\end{center}
\end{frame}


\subsection{Subtraktion mit Hilfe des Zweierkomplements}
\begin{frame} 		
	\frametitle{Subtraktion mit Hilfe des Zweierkomplements}
	Da, in der Digitaltechnik, für die Subtraktion von Dualzahlen keine logische Verknüpfung existiert, ist man gezwungen eine
	Subtraktion in eine Addition umwandeln. \pause
		\[ \begin{array}{rcl}
   				 2 - 6       & = & (-4)			\\ \pause 
  			         2 + (-6)   & = & (-4)
		\end{array} \]
\end{frame}


\begin{frame} 		
	\frametitle{Subtraktion mit Hilfe des Zweierkomplements}
		\hspace*{5mm} $ \underline{2 - 6 = \, ? }$ \\ \pause 
		1. Schritt: In eine Dualzahl wandeln:\\  
		\hspace*{5mm}  $ 2 - 6 \Rightarrow 10 - 110$\\ \pause
		2. Schritt: Stellen auffüllen:\\ 
		\hspace*{5mm} $ 0010 - 0110 =$ $ ? $\\\pause 
		3. Schritt: Bits negieren:\\
		\hspace*{5mm} $ 0110 \Rightarrow 1001$\\ \pause 
		4. Schritt: Hinzuaddieren von 1:\\ 
		\hspace*{5mm} $ 1001 + 0001 = 1010$\\ \pause
		5. Schritt: Minuend und Zweierkomplement addieren:\\ 
		\hspace*{5mm} $ 0010 + 1010 = 1100$\\ \pause
		6. Schritt: Ergebnis negieren:\\ 
		\hspace*{5mm} $ 100 \Rightarrow 011$\\ 
		\end{frame}
\begin{frame} 		
	\frametitle{Subtraktion mit Hilfe des Zweierkomplements}
		7. Schritt: Hinzuaddieren von 1:\\
		\hspace*{5mm} $ 011 + 001 = 100$\\ \pause 
		8. Schritt: In eine Dezimalzahl wandeln:\\ 
		\hspace*{5mm} $ 100 \Rightarrow 4$ ; da das höchstwertige Bit 1 ist: Endergebnis $=$ $ -4$\\
		9. Schritt: Ergebniss:\\ 
		\hspace*{5mm} $ \underline{2 - 6 = \, (-4) }$
\end{frame}


\subsection{Multiplikation von Dualzahlen}
\begin{frame} 		
	\frametitle{Multiplikation von Dualzahlen}
	Bei der binären Multiplikation werden Produkte mit den einzelnen Stellen des Multiplikators gebildet und anschließend
	Stellenrichtig addiert. \\
	Da die Stellen des Multiplikators nur die Zahlenwerte Null und Eins annehmen können, muss der 
	Multiplikand nur mit Null und Eins multipliziert werden. Dies kann mit einer einfachen UND-Verknüpfung gelößt werden.
\end{frame}


\begin{frame} 		
	\frametitle{Multiplikation von Dualzahlen}
	Rechenbeispiel für eine Multiplikation mit Dual-Zahlen.
	\begin{center}
		\begin{tabular}{crcl}
				&		 	       	  1 0 1 1 $\times$ 1 0 1 0	 	  	\\ \hline
				&							1011000			\\
				&							000000			\\
				&							10110			\\
		 +		&							0000			  	\\ \hline
				&						        1101110		  
		\end{tabular}
	\end{center}
\end{frame}



%\section{Umsetzung in eine digitale Rechenschaltung} 
%\subsection{Addierer}
%\subsubsection{Halbaddierer}
%\begin{frame}
%	\frametitle{Halbaddierer}
%	\begin{columns}
%		\begin{column}{5cm}
%			Ein Halbaddierer besitzt zwei Ein-, und zwei Ausgänge. An die Eingänge \emph{x} und \emph{y} werden jeweils 
%			die Ziffern angelegt die man addieren möchte. An dem ersten Ausgang liegt die Summe $s$ der Addition an, am 
%			zweiten Ausgang der Übertrag $c$.
%			\end{column}
%	\begin{column}{5cm}
%	
%	\begin{tabular}{|c|c|}
%		\hline
%		\textbf{Kursleiter} & \textbf{Titel} \\
%		\hline
%		Sascha Frank &  \LaTeX \ Kurs 1 \\
%		\hline	
%		Sascha Frank & \LaTeX \ Kursreihe \\
%		\hline
%	\end{tabular}
%\end{frame}






\begin{frame}\frametitle{Aufz\"ahlung mit Pausen}
\begin{itemize}
\item  Einf\"uhrungskurs in \LaTeX 
\item  Kurs 2 \pause 
\item  Seminararbeiten und Pr\"asentationen mit \LaTeX \pause 
\item  Die Beamerclass
\end{itemize} 
\end{frame}

\subsection{Listen II}
\begin{frame}\frametitle{Numerierte Liste}
\begin{enumerate}
\item  Einf\"uhrungskurs in \LaTeX 
\item  Kurs 2
\item  Seminararbeiten und Pr\"asentationen mit \LaTeX 
\item  Die Beamerclass
\end{enumerate}
\end{frame}
\begin{frame}\frametitle{Numerierte Liste mit Pausen}
\begin{enumerate}
\item  Einf\"uhrungskurs in \LaTeX \pause 
\item  Kurs 2 \pause 
\item  Seminararbeiten und Pr\"asentationen mit \LaTeX \pause 
\item  Die Beamerclass
\end{enumerate}
\end{frame}

\section{Abschnitt Nr.3} 
\subsection{Tabellen}
\begin{frame}\frametitle{Tabellen}
\begin{tabular}{|c|c|c|}
\hline
\textbf{Zeitpunkt} & \textbf{Kursleiter} & \textbf{Titel} \\
\hline
WS 04/05 & Sascha Frank &  Erste Schritte mit \LaTeX  \\
\hline
SS 05 & Sascha Frank & \LaTeX \ Kursreihe \\
\hline
\end{tabular}
\end{frame}


\begin{frame}\frametitle{Tabellen mit Pause}
\begin{tabular}{c c c}
A & B & C \\ 
\pause 
1 & 2 & 3 \\  
\pause 
A & B & C \\ 
\end{tabular} 
\end{frame}


\section{Abschnitt Nr. 4}
\subsection{Bl\"ocke}
\begin{frame}\frametitle{Bl\"ocke}

\begin{block}{Blocktitel}
Blocktext 
\end{block}

\begin{exampleblock}{Blocktitel}
Blocktext 
\end{exampleblock}


\begin{alertblock}{Blocktitel}
Blocktext 
\end{alertblock}
\end{frame}

\section{Abschnitt Nr. 5}
\subsection{Geteilter Bildschirm}

\begin{frame}\frametitle{Zerteilen des Bildschirmes}
\begin{columns}
\begin{column}{5cm}
\begin{itemize}
\item Beamer 
\item Beamer Class 
\item Beamer Class Latex 
\end{itemize}
\end{column}
\begin{column}{5cm}
\begin{tabular}{|c|c|}
\hline
\textbf{Kursleiter} & \textbf{Titel} \\
\hline
Sascha Frank &  \LaTeX \ Kurs 1 \\
\hline
Sascha Frank & \LaTeX \ Kursreihe \\
\hline
\end{tabular}
\end{column}
\end{columns}
\end{frame}










\end{document}