\RequirePackage{etex}
\documentclass{beamer}
\usepackage[latin9]{inputenc}
\usepackage[ngerman]{babel}
\usepackage{amsmath}
\usepackage{listings}
\usepackage{color}
\usepackage{pictex}

\title[Rechenwerk]{Digitales Rechenwerk}
\author{Dominik Eisele}
\institute[WSS]{Werner-Siemens-Schule}
\date{\today}
\subject{FTP}
\keywords{FTP, File Transfer Protocol}

\usetheme{Singapore}
\usecolortheme{rose}
\usefonttheme{serif}
\setbeamertemplate{navigation symbols}{}
\setbeamertemplate{footline}[frame number]
\beamersetuncovermixins{\opaqueness<1>{25}}{\opaqueness<2->{15}}

\definecolor{dkgreen}{rgb}{0,0.6,0}
\definecolor{gray}{rgb}{0.5,0.5,0.5}
\definecolor{mauve}{rgb}{0.58,0,0.82}

\lstset{language=C}
\lstset{numbers=left,
	numberstyle=\tiny,
	numbersep=5pt,
	breaklines=true,
	showstringspaces=false,
	frame=l ,
	xleftmargin=15pt,
	xrightmargin=15pt,
	basicstyle=\ttfamily\scriptsize,
	stepnumber=1,
	keywordstyle=\color{blue},		% keyword style
	commentstyle=\color{dkgreen},	% comment style
	stringstyle=\color{mauve}		% string literal style
}

%\AtBeginDocument{\addtobeamertemplate{block begin}{\setlength\abovedisplayskip{0pt}}}
%\AtBeginDocument{\addtobeamertemplate{block begin}{\setlength\abovedisplayshortskip{0pt}}}

\let\oldsqrt\sqrt 
\def\sqrt{\mathpalette\DHLhksqrt}
\def\DHLhksqrt#1#2{\setbox0=\hbox{$#1\oldsqrt{#2\,}$}\dimen0=\ht0
\advance\dimen0-0.3\ht0
%0.3 ist das Ma� f�r die Hakenl�nge, relativ zum Inhalt der Wurzel
\setbox2=\hbox{\vrule height\ht0 depth -\dimen0}%
{\box0\lower0.4pt\box2}}

\setcounter{tocdepth}{1} 

\begin{document}

\begin{frame}
	\titlepage
\end{frame}

\begin{frame}{Inhalt}
	\tableofcontents
\end{frame}


\section{Rechnen mit dualen Zahlen}
\subsection{Rechnen mit dualen Zahlen-dots}

\begin{frame}{Rechnen mit Dualzahlen} 
	Das Rechnen mit Dualzahlen verl�uft nach den selben Rechenregeln wie das Rechnen mit Dezimalzahlen.
\end{frame}

\begin{frame}{Addition von Dualzahlen}
	Rechenbeispiel f�r eine Addition mit Dual-Zahlen.
	\begin{center}
		\begin{tabular}{crcl}
				&		 	       	 1 1 1 0 1 0 0 1 0 		 	  	\\
		 +  		&				 0 0 1 1 1 0 1 0 0			  	\\\hline
				&			      1 0 0 1 0 0 0 1 1 0			  
		\end{tabular}
	\end{center}
\end{frame}


\begin{frame}{Subtraktion von Dualzahlen}
	Rechenbeispiel f�r eine Subtraktion mit Dual-Zahlen.
	\begin{center}
		\begin{tabular}{crcl}
				&		 	       	  1 1 1 0 1 0 0 1 0 		 	  	\\
		 -  		&				  0 0 1 1 1 0 1 0 0			  	\\\hline
				&			          1 0 1 0 1 1 1 1 0			  
		\end{tabular}
	\end{center}
\end{frame}


\begin{frame}{Subtraktion mit Hilfe des Zweierkomplements}
	Da, in der Digitaltechnik, f�r die Subtraktion von Dualzahlen keine logische Verkn�pfung existiert, ist man gezwungen eine
	Subtraktion in eine Addition umwandeln.
		
	\begin{align*}
		\onslide<2->{2 - 6 &= (-4) \\} 
		\onslide<3->{2 + \left(-6\right) &= (-4) \\}  
	\end{align*}
\end{frame}


\begin{frame} 		
	\frametitle{Subtraktion mit Hilfe des Zweierkomplements}
		\hspace*{5mm} $ \underline{2 - 6 = \, ? }$ \\ \pause 
		1. Schritt: In eine Dualzahl wandeln:\\  
		\hspace*{5mm}  $ 2 - 6 \Rightarrow 10 - 110$\\ \pause
		2. Schritt: Stellen auff�llen:\\ 
		\hspace*{5mm} $ 0010 - 0110 =$ $ ? $\\\pause 
		3. Schritt: Bits negieren:\\
		\hspace*{5mm} $ 0110 \Rightarrow 1001$\\ \pause 
		4. Schritt: Hinzuaddieren von 1:\\ 
		\hspace*{5mm} $ 1001 + 0001 = 1010$\\ \pause
		5. Schritt: Minuend und Zweierkomplement addieren:\\ 
		\hspace*{5mm} $ 0010 + 1010 = 1100$\\ \pause
		6. Schritt: Ergebnis negieren:\\ 
		\hspace*{5mm} $ 100 \Rightarrow 011$\\ 
		\end{frame}
\begin{frame} 		
	\frametitle{Subtraktion mit Hilfe des Zweierkomplements}
		7. Schritt: Hinzuaddieren von 1:\\
		\hspace*{5mm} $ 011 + 001 = 100$\\ \pause 
		8. Schritt: In eine Dezimalzahl wandeln:\\ 
		\hspace*{5mm} $ 100 \Rightarrow 4$ ; da das h�chstwertige Bit 1 ist: Endergebnis $=$ $ -4$\\ \pause
		9. Schritt: Ergebniss:\\ 
		\hspace*{5mm} $ \underline{2 - 6 = \, (-4) }$
\end{frame}


\begin{frame}{Multiplikation von Dualzahlen}
	Bei der bin�ren Multiplikation werden Produkte mit den einzelnen Stellen des Multiplikators gebildet und anschlie�end
	Stellenrichtig addiert. \\
	Da die Stellen des Multiplikators nur die Zahlenwerte Null und Eins annehmen k�nnen, muss der 
	Multiplikand nur mit Null und Eins multipliziert werden. Dies kann mit einer einfachen UND-Verkn�pfung gel��t werden.
\end{frame}


\begin{frame}{Multiplikation von Dualzahlen}
	Rechenbeispiel f�r eine Multiplikation mit Dual-Zahlen.
	\begin{center}
		\begin{tabular}{crcl}
				&		 	       	  1 0 1 1 $\times$ 1 0 1 0	 	  	\\ \hline
				&							1011000			\\
				&							000000			\\
				&							10110			\\
		 +		&							0000			  	\\ \hline
				&						        1101110		  
		\end{tabular}
	\end{center}
\end{frame}

\section{Rechenwerk}
\subsection{Umsetzung-dots}
\begin{frame}{Halbaddierer}
	\begin{columns}
		\begin{column}{0.35\linewidth}
			Ein Halbaddierer besitzt zwei Ein-, und zwei Ausg�nge. An die Eing�nge \emph{x} und \emph{y} werden jeweils die 
			Ziffern angelegt die man addieren m�chte. An dem ersten Ausgang liegt die Summe $s$ der Addition an,
			am zweiten Ausgang der �bertrag $c$.
		\end{column}
		
		\begin{column}{0.65\linewidth}
			\begin{table}[h]
				\center
				\begin{tabular}{c|c|c|c}
					\ \textbf{x} \ 	& \ \textbf{y} \		& \ \textbf{�bertrag c} \ & \ \textbf{Summe s} \ 	 	\\ \hline
					0 	& 0 		& 0          			& 0       			\\ \hline
					0 	& 1 		& 0          			& 1       			\\ \hline
					1 	& 0		& 0          			& 1      			 \\ \hline
					1	& 1 		& 1          			& 0      			 \\
				\end{tabular}
				\label{tab:halbaddierer}
			\end{table}
		\end{column}
	\end{columns}
\end{frame}


\begin{frame}{Schaltbild Halbaddierer}	
	\begin{columns}
		\begin{column}{0.5\linewidth}
			In Schaltungen wird der Halbaddierer aus zwei Bauteilen zusammengesetzt, ein Exklusiv-ODER (XOR) und ein
			UND (AND).
		\end{column}
		
		\begin{column}{0.5\linewidth}
			\begin{figure}
				\center
				\input skizze_halbaddierer
				\label{fig:halbaddierer}
			\end{figure}
		\end{column}
	\end{columns}
\end{frame}


\begin{frame}{Volladdierer}
	\begin{columns}
		\begin{column}{0.52\linewidth}
			Der Volladdierer besteht aus zwei Halbaddierern und einem ODER. Da ein Volladdierer einen zus�tzlichen Eingang
			($c_{in}$) hat, kann man mit ihm den �bertrag aus einer vohergegangenen Addition mit in die Rechnung
			einbeziehen. Man kann somit mehrere Volladdierer hintereinander schalten um gr��ere Zahlen miteinander zu addieren.
			Dabei verbindet man den Carry out Ausgang mit dem Carry in Ausgang des h�herwertigen Volladierers.
		\end{column}
		
		\begin{column}{0.48\linewidth}
			\begin{table}
				\begin{tabular}{c|c|c|c|c}
					\ $\mathbf{x}$ \ & \ $\mathbf{y}$ \ & \ $\mathbf{c_{in}}$ \ & \ $\mathbf{c_{out}}$ \ & \ $\mathbf{s}$ \ \\ \hline
					0          & 0          & 0            & 0             & 0                              \\ \hline
					0          & 0          & 1            & 0             & 1                              \\ \hline
					0          & 1          & 0            & 0             & 1                              \\ \hline
					0          & 1          & 1            & 1             & 0                              \\ \hline
					1          & 0          & 0            & 0             & 1                              \\ \hline
					1          & 0          & 1            & 1             & 0                              \\ \hline
					1          & 1          & 0            & 1             & 0                              \\ \hline
					1          & 1          & 1            & 1             & 1
				\end{tabular}
				\label{tab:volladdierer}
			\end{table}
		\end{column}
	\end{columns}
\end{frame}


\begin{frame}{Schaltbild Volladdierer}	
	In Schaltungen wird der Halbaddierer aus zwei Bauteilen zusammengesetzt, ein Exklusiv-ODER (XOR) und ein UND (AND).
	\begin{figure}
		\center
		\input skizze_volladdierer
		\label{fig:volladdierer}
	\end{figure}
\end{frame}


\section{Quellen}
\subsection{Quellen-dots}
\begin{frame}
\end{frame}

\end{document}
